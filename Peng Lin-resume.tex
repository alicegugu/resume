% Created 2014-12-01 Mon 02:05
\documentclass[10pt, a4paper]{article}
\usepackage[T1]{fontenc}
\usepackage{fixltx2e}
\usepackage{graphicx}
\usepackage{longtable}
\usepackage{float}
\usepackage{wrapfig}
\usepackage{rotating}
\usepackage[normalem]{ulem}
\usepackage{amsmath}
\usepackage{textcomp}
\usepackage{marvosym}
\usepackage{wasysym}
\usepackage{amssymb}
\usepackage{hyperref}
\tolerance=1000
\usepackage{tabu}
\usepackage[margin=0.75in]{geometry}
\usepackage{tabularx}
\usepackage{sectsty}
\usepackage{color}
\definecolor{linkcolour}{rgb}{0,0.2,0.6}
\hypersetup{colorlinks,breaklinks, urlcolor=linkcolour,linkcolor=linkcolour}
\author{Peng Lin}
\date{\today}
\title{}
\hypersetup{
  pdfkeywords={},
  pdfsubject={},
  pdfcreator={Emacs 24.4.1 (Org mode 8.2.10)}}
\begin{document}

\pagestyle{empty}
\setlength{\parindent}{0cm}

\sectionfont{
  \huge\bfseries
}

\subsectionfont{
  \sectionrule{0pt}{0pt}{-5pt}{0.8pt}
}

\section*{Peng Lin}
\label{sec-1}
\begin{center}
\begin{tabularx}{\linewidth}{Xrl}
Building No. 3, 399 Keyuan Rd & E-mail: & \href{mailto:penglin03@gmail.com}{penglin03@gmail.com}\\
Pudong, Shanghai, 201203, China & Phone: & +86 189 1852 2776\\
 & Homepage: & \url{https://sites.google.com/site/penglin03/}\\
\end{tabularx}
\end{center}

\subsection*{Education}
\label{sec-1-1}
\begin{center}
\begin{tabularx}{\linewidth}{lXr}
2009.9-2012.3 & \textbf{Shanghai Jiao Tong University} & \textbf{SJTU}\\
 & - \emph{Institute of Image Processing \& Pattern Recognition} & \\
 & Degree: Master of Engineering in Control Theory \& Control Engineering & \\
 & Research fields: Computer Vision \& Machine Learning & \\
 & Advisor:  {\href{http://automation.sjtu.edu.cn/en/ShowPeople.aspx?info_id=406&info_lb=326&flag=224}{Yuming Zhao}} \& Fuqiao Hu & \\
 & Thesis: Body Part Recognition Based on Depth Images by Learning & \\
 & Score (GPA): Overall: 3.3/4.3, Major: 3.5/4.3 & \\
\end{tabularx}
\end{center}

\begin{center}
\begin{tabularx}{\linewidth}{lXr}
2005.9-2009.6 & \textbf{Shanghai Jiao Tong University} & \textbf{SJTU}\\
 & - \emph{School of Electronic, Information and Electrical Engineering} ({\href{http://english.seiee.sjtu.edu.cn/}{SEIEE}}) & \\
 & Degree: Bachelor of Engineering in Automation & \\
 & Thesis: The Analysis and Design of Transfer Function in the Three-Dimensional Volume Rendering & \\
 & Score (GPA): Overall: 3.4/4.3, Major: 3.5/4.3 & \\
\end{tabularx}
\end{center}

\subsection*{Work Experience}
\label{sec-1-2}
\begin{center}
\begin{tabularx}{\linewidth}{lXr}
2012.4-present & \textbf{Software Engineer at Marvell Technology Group Ltd.} & Shanghai, China\\
 & Focus: Image Processing \& Camera Calibration for mobile camera & \\
2011.5-2011.11 & \textbf{Software Intern at Intel Corporation Ltd.} & Shanghai, China\\
 & Focus: System Administrator (Linux \& Windows servers on Xen) & \\
2010.7-2010.10 & \textbf{Software Intern at Omron Corporation Ltd.} & Shanghai, China\\
 & Focus: Designer for Omron's exhibition at World Expo 2010 Shanghai & \\
\end{tabularx}
\end{center}


\subsection*{Publications}
\label{sec-1-3}
\begin{itemize}
\item \textbf{Peng Lin}, Chao Zhang, Zhuliang Li, Yuming Zhao, Human Body Part
Recognition Based on Depth Image Learning[J], Computer Engineering, Vol.38
(16), pp.185-188, 2012, DOI: 10.3969/j.issn.1000-3428.2012.16.048. (In
Chinese) [\href{https://drive.google.com/viewerng/viewer?a=v&pid=sites&srcid=ZGVmYXVsdGRvbWFpbnxwZW5nbGluMDN8Z3g6MmZhYzU4NjM1NDlkMjg1Mw}{PDF}]
\item Ling Cai, \textbf{Peng Lin}, Yuming Zhao, Chenghua Wang, Texture Image
Segmentation by Active Bayesian Contour, International Conference on
System Design and Data Processing (ICSDDP), Taiyuan, Shanxi, pp.357-360,
2011.2. [\href{https://drive.google.com/viewerng/viewer?a=v&pid=sites&srcid=ZGVmYXVsdGRvbWFpbnxwZW5nbGluMDN8Z3g6NzYwYzEzZTViNGRmY2NjYg}{PDF}]
\end{itemize}


\subsection*{Research \& Projects}
\label{sec-1-4}
\begin{longtabu} spread \linewidth {lX}
2011.03-2012.03 & \textbf{Human body part recognition}: \emph{Master's project at} SJTU. The motivation of this project was to do fast segmentation and human body classification by random forest. I investigated and implemented a human body part recognition algorithm, by exploiting the techniques of depth imaging and ensemble learning. Compared with other geometry based methods, the algorithm used less locality information but gained a real time classification. (Advisor: Yuming Zhao)\\
2014.04-present & \textbf{Color non-uniformity correction}: \emph{Current project}. The Color Non-Uniformity (Shading) Correction is an intractable problem in mobile camera. I am researching and designing a vectorization based method, which compresses features from massive white charts and searches the best fitting curve to minimize entropy. To embedded systems, I am building up a minimal matrix library with numerical stability and accuracy. Sample code: {\href{https://github.com/penglin03/Numic}{Numic}}, ({\href{https://jhupbooks.press.jhu.edu/content/matrix-computations-0}{ref: Matrix Computations 4th}})\\
2013.07-2014.03 & \textbf{Camera calibration \& tuning}: I tuned a hundred of parameters for the algorithms to reach best image quality. I was researching {\href{http://www.amazon.com/Introduction-Color-Imaging-Science-Hsien-Che/dp/0521103134}{Color Science}} and building up standard workflows of tests. The tests could attribute to the algorithms, which caused the defect, like whether sharpening or de-noise caused texture loss.\\
2012.10-2013.06 & \textbf{Algorithms on image processor}: This was a prospective study aimed to test the Image Processor's parallel capability in complex algorithms. I implemented parallel algorithms in Python, which included format conversion, image filtering, feature extraction and texture processing.\\
2012.07-2012.09 & \textbf{Panorama optimization}: The hotspot was in mosaic and pyramid blending. I searched and employed the linear interpolation to replace default non-linear functions, combined with LUT and fixed point computing. On 1GHz CPUs, the time cost was reduced from 20s to 10s per frame.\\
\end{longtabu}

\subsection*{Selected Courses}
\label{sec-1-5}
\begin{center}
\begin{tabularx}{\linewidth}{p{0.6in}X}
Master & Pattern Recognition (A), Artificial Neural Network (A), Principle and System of Intelligence (A-)\\
Bachelor & Signals and Systems (A-), Mathematical Analysis II (A-), Discrete Mathematics (A), Physics I (A-)\\
Online & {\href{https://www.coursera.org/course/ml}{Machine Learning}}, Coursera 2014, ({\href{https://drive.google.com/viewerng/viewer?a=v&pid=sites&srcid=ZGVmYXVsdGRvbWFpbnxwZW5nbGluMDN8Z3g6MmNkZTRiNWZkNmQ5MzRm}{record}}); {\href{https://www.coursera.org/course/compphoto}{Computational Photography}}, Coursera 2013, ({\href{https://drive.google.com/viewerng/viewer?a=v&pid=sites&srcid=ZGVmYXVsdGRvbWFpbnxwZW5nbGluMDN8Z3g6OTM5OTQ4OWFjNTI4MjJj}{record}}); {\href{http://ocw.mit.edu/courses/electrical-engineering-and-computer-science/6-001-structure-and-interpretation-of-computer-programs-spring-2005/}{SICP}}, MIT OCW 2013-present\\
\end{tabularx}
\end{center}

\subsection*{Honors \& Awards}
\label{sec-1-6}
\begin{center}
\begin{tabularx}{\linewidth}{lXr}
2011-2012 & Graduate Fellowship & SJTU\\
2008-2009 & National Encouragement Scholarship & {\href{http://www.moe.edu.cn/publicfiles/business/htmlfiles/moe/moe_2792/index.html}{MoE, China}}\\
2007-2008 & Scholarship of Rockwell & Rockwell\\
 & - rewarding 3/108 students' major GPA in junior year & \\
2006-2007 & Merit Student of the Year & SJTU\\
 & - 1 each class (1/29), for the overall performance of the year & \\
2006-2007 & Academic Excellence Scholarship, Third Class & SJTU\\
2003-2004 & First Prize in National Mathematical Olympiad Competition & {\href{http://www.cms.org.cn/cms-e/index.html}{CMS, China}}\\
 & - one of top 30 high school students at all grade levels in my province & \\
 & - offered admission to SJTU waived of National Matriculation Examination 2005 & \\
\end{tabularx}
\end{center}


\subsection*{Tests}
\label{sec-1-7}
\begin{center}
\begin{tabularx}{\linewidth}{p{0.8in}X}
2014.07.06 & TOEFL 94\\
2014.11.01 & GRE   V150, Q170, AW3.0\\
\end{tabularx}
\end{center}

\subsection*{Teaching Experience}
\label{sec-1-8}
\begin{center}
\begin{tabularx}{\linewidth}{lX}
2009.09-2011.06 & Undergraduate Class Teacher. I had served and managed a class of 27 students for two years.\\
2010.09-2011.01 & Teaching Assistantship. Course AU311: Introductory Pattern Recognition, School of Electronic, Information and Electrical Engineering (SEIEE), Shanghai Jiao Tong University, Fall, 2010.\\
\end{tabularx}
\end{center}

\subsection*{Technical Skills}
\label{sec-1-9}
\begin{center}
\begin{tabularx}{\linewidth}{p{1in}X}
Programming languages & C, Python, Octave/Matlab, C++, Lisp, Java, C\# with practical experience.\\
 & \\
Favorite Tools \& Libraries & Linux, Emacs (Org mode), Vi, GCC, Bash, Git, Html, CSS, JSON, Xml (libxml2), OpenCV, OpenGL (freeglut), \LaTeX{},\\
 & \\
Specialized Knowledge & Fundamental Machine Learning \& Image Processing Algorithms, Random Forest, Neural Network, PCA, Depth Imaging, Image Segmentation, Object Tracking, Image Quality, Color Imaging Science, Embedded System\\
\end{tabularx}
\end{center}
% Emacs 24.4.1 (Org mode 8.2.10)
\end{document}